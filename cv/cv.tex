\documentclass{ali-resume}

\author{Alistair Pattison}
\title{
{\it (651) 706-6713}
\ | \
{\it pattisona@carleton.edu}
\ | \
\href{http://alipatti.com}{\it alipatti.com}
}

\addbibresource{publications.bib}

\begin{document}

\maketitle

\section{Education}

\resumeitem{Bachelors of Arts, Mathematics \& Statistics}
{Carleton College}
{September 2020 -- June 2024}

\begin{itemize}
	\item 3.99 GPA, captain of varsity track and field, minor in Digital Arts and Humanities.
	\item Studied abroad at the University of Cambridge taking classes in computer science and history.
	      % \item Math coursework in algebra, analysis, combinatorics, and analytic number theory.
	      % \item Computer science coursework in algorithms, cryptography, artificial intelligence, and quantum computing.
	      % \item Statistics coursework in probability, regression, time-series analysis, data science, and data visualization.
\end{itemize}

% \resumeitem{Talented Youth Mathematics Program}
% {University of Minnesota}
% {2015 -- 2020}
%
% \begin{itemize}
% 	\item Completed selective high-school math program culminating in six semesters of set theory, vector calculus, linear algebra, and differential equations.
% \end{itemize}

\section{Work Experience}

\resumeitem{Statistical Consultant}{Minnesota Department of Human Services}{September 2023 -- present}

\begin{itemize}
	\item Consult for Age-Friendly Minnesota, a collaboration between several government and private entities with the goal of making the state more accessible to and inclusive of older Minnesotans.
	\item Analyze a large survey data set to identify disparities in wealth, housing, and internet access among elderly Minnesotans.
	\item Create data visualizations and give bi-weekly reports to a group of government employees and other consultants.
\end{itemize}

\resumeitem{Research Assistant}
{University of Minnesota}
{September 2022 -- present}

\begin{itemize}
	\item Developed a method for reporting abuse on encrypted messaging platforms that provably maintains privacy until certain thresholds are met, e.g., if a user is reported enough times or by enough people.
	\item Wrote and benchmarked a proof-of-concept implementation in Rust.
	\item Presented findings at a first-tier security conference. We plan to publish more results in 2024.
\end{itemize}

\resumeitem{Teaching Assistant}
{Carleton College}
{March 2021 -- present}

\begin{itemize}
	\item Tutor students, grade homework, and hold office hours for classes in the math and computer science departments.
	\item Classes include
	      algorithms, % (CS 252),
	      real analysis, % (MATH 321),
	      abstract algebra, % (MATH 342),
	      data visualization, % (CS 314),
	      artificial intelligence, % (CS 321),
	      and calculus. % (MATH 120),
	      %   and math structures. % (MATH 236).
\end{itemize}

\resumeitem{The Food Guy}
{YMCA Camp Widjiwagan}
{June -- September 2021}

\begin{itemize}
	\item Oversaw the purchase and distribution of food for nearly 200 wilderness trips across the US and Canada.
	\item Managed a \$95,000 budget alongside two other employees and taught basic nutrition to kids aged 12 to 18.
	\item Lead a week-long canoeing trip with BOLD, a program dedicated to making the outdoors more diverse and accessible.
\end{itemize}

\resumeitem{Dining Hall Worker}
{Carleton College}
{September 2020 -- March 2021}

% \resumeitem{Farm Hand}
% {Hansen Tree Farm}
% {2018 -- 2021}

% \resumeitem{Owner}
% {Pattison Lawn Care}
% {2015 -- 2020}

\null

\section{Publications}

% TODO add url: (will appear at https://arxiv.org/user/)
\fullcite{cerberus}

\fullcite{threshold-reporting}

\section{Awards}

\newcommand{\award}[2]{\textbf{#1}, \textit{#2}}

\begin{itemize}[label={}, leftmargin=0em]
	\item \award{Phi Beta Kappa}{2023}
	\item \award{Dean's List}{2023, 2022, 2021}
	\item \award{James F. Koehler Scholarship}{2022}
	\item \award{National Merit Scholarship}{2020}
\end{itemize}

\section{Projects}

\newcommand{\project}[2]{%
	\subsection{#1
		\hfill
		\normalfont \small \faicon{github}
		\href{http://github.com/alipatti/#2}{\textit{alipatti/#2}}
	}}

\project{Ole or Carl?}{oleorcarl}

\begin{itemize}
	\item This project tests the validity of a long-standing stereotype that students from Northfield's two colleges look meaningfully different. If so, then a machine learning model should be able to tell them apart? Right?
	\item Uses an SVM classifier on embeddings of student's faces taken from directory photos.
\end{itemize}

\project{Carleton Course Classifier}{course-classifier}

\begin{itemize}
	\item A Tensorflow language model that predicts a course's department and number given its description.
	\item Achieves 64\% accuracy in a 54-class problem using a neural net and the Stanford GloVe embeddings.
\end{itemize}

\project{Wordle Bot}{wordle-bot}

\begin{itemize}
	\item A NumPy implementation, of 3Blue1Brown's information-theoretic Wordle bot, with a few performance improvements.
\end{itemize}

\project{In Passing}{in-passing}

\begin{itemize}
	\item A visualization of passing and touch data from the 2022 FIFA Men's World Cup, built with d3.js.
\end{itemize}

\section{Skills}

\textbf{Programming:} Python, R, JavaScript, \LaTeX, Rust, Java, SQL, git, Docker.

\textbf{Web Development:} React, Typescript, Next.js, Prisma.

\textbf{Data Science:} NumPy, dplyr, pandas, Matplotlib, ggplot2, tidyverse, SQL.

\textbf{Machine Learning:} TensorFlow, vector embeddings, SVMs, scikit-learn.

\textbf{Data Visualization:} ggplot2, matplotlib, d3.js.

% \textbf{Writing:} adept at writing for many different audiences.

\end{document}
